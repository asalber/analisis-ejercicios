% Options for packages loaded elsewhere
\PassOptionsToPackage{unicode}{hyperref}
\PassOptionsToPackage{hyphens}{url}
\PassOptionsToPackage{dvipsnames,svgnames,x11names}{xcolor}
%
\documentclass[
  letterpaper,
  DIV=11,
  numbers=noendperiod]{scrreprt}

\usepackage{amsmath,amssymb}
\usepackage{lmodern}
\usepackage{iftex}
\ifPDFTeX
  \usepackage[T1]{fontenc}
  \usepackage[utf8]{inputenc}
  \usepackage{textcomp} % provide euro and other symbols
\else % if luatex or xetex
  \usepackage{unicode-math}
  \defaultfontfeatures{Scale=MatchLowercase}
  \defaultfontfeatures[\rmfamily]{Ligatures=TeX,Scale=1}
  \setmainfont[Numbers=Lowercase,Numbers=Proportional]{TeX Gyre Pagella}
\fi
% Use upquote if available, for straight quotes in verbatim environments
\IfFileExists{upquote.sty}{\usepackage{upquote}}{}
\IfFileExists{microtype.sty}{% use microtype if available
  \usepackage[]{microtype}
  \UseMicrotypeSet[protrusion]{basicmath} % disable protrusion for tt fonts
}{}
\makeatletter
\@ifundefined{KOMAClassName}{% if non-KOMA class
  \IfFileExists{parskip.sty}{%
    \usepackage{parskip}
  }{% else
    \setlength{\parindent}{0pt}
    \setlength{\parskip}{6pt plus 2pt minus 1pt}}
}{% if KOMA class
  \KOMAoptions{parskip=half}}
\makeatother
\usepackage{xcolor}
\setlength{\emergencystretch}{3em} % prevent overfull lines
\setcounter{secnumdepth}{5}
% Make \paragraph and \subparagraph free-standing
\ifx\paragraph\undefined\else
  \let\oldparagraph\paragraph
  \renewcommand{\paragraph}[1]{\oldparagraph{#1}\mbox{}}
\fi
\ifx\subparagraph\undefined\else
  \let\oldsubparagraph\subparagraph
  \renewcommand{\subparagraph}[1]{\oldsubparagraph{#1}\mbox{}}
\fi


\providecommand{\tightlist}{%
  \setlength{\itemsep}{0pt}\setlength{\parskip}{0pt}}\usepackage{longtable,booktabs,array}
\usepackage{calc} % for calculating minipage widths
% Correct order of tables after \paragraph or \subparagraph
\usepackage{etoolbox}
\makeatletter
\patchcmd\longtable{\par}{\if@noskipsec\mbox{}\fi\par}{}{}
\makeatother
% Allow footnotes in longtable head/foot
\IfFileExists{footnotehyper.sty}{\usepackage{footnotehyper}}{\usepackage{footnote}}
\makesavenoteenv{longtable}
\usepackage{graphicx}
\makeatletter
\def\maxwidth{\ifdim\Gin@nat@width>\linewidth\linewidth\else\Gin@nat@width\fi}
\def\maxheight{\ifdim\Gin@nat@height>\textheight\textheight\else\Gin@nat@height\fi}
\makeatother
% Scale images if necessary, so that they will not overflow the page
% margins by default, and it is still possible to overwrite the defaults
% using explicit options in \includegraphics[width, height, ...]{}
\setkeys{Gin}{width=\maxwidth,height=\maxheight,keepaspectratio}
% Set default figure placement to htbp
\makeatletter
\def\fps@figure{htbp}
\makeatother

\usepackage{venndiagram}
\KOMAoption{captions}{tableheading}
\makeatletter
\@ifpackageloaded{tcolorbox}{}{\usepackage[many]{tcolorbox}}
\@ifpackageloaded{fontawesome5}{}{\usepackage{fontawesome5}}
\definecolor{quarto-callout-color}{HTML}{909090}
\definecolor{quarto-callout-note-color}{HTML}{0758E5}
\definecolor{quarto-callout-important-color}{HTML}{CC1914}
\definecolor{quarto-callout-warning-color}{HTML}{EB9113}
\definecolor{quarto-callout-tip-color}{HTML}{00A047}
\definecolor{quarto-callout-caution-color}{HTML}{FC5300}
\definecolor{quarto-callout-color-frame}{HTML}{acacac}
\definecolor{quarto-callout-note-color-frame}{HTML}{4582ec}
\definecolor{quarto-callout-important-color-frame}{HTML}{d9534f}
\definecolor{quarto-callout-warning-color-frame}{HTML}{f0ad4e}
\definecolor{quarto-callout-tip-color-frame}{HTML}{02b875}
\definecolor{quarto-callout-caution-color-frame}{HTML}{fd7e14}
\makeatother
\makeatletter
\@ifpackageloaded{tikz}{}{\usepackage{tikz}}
\makeatother
\makeatletter
\@ifpackageloaded{bookmark}{}{\usepackage{bookmark}}
\makeatother
\makeatletter
\@ifpackageloaded{caption}{}{\usepackage{caption}}
\AtBeginDocument{%
\ifdefined\contentsname
  \renewcommand*\contentsname{Indice de contenidos}
\else
  \newcommand\contentsname{Indice de contenidos}
\fi
\ifdefined\listfigurename
  \renewcommand*\listfigurename{Listado de Figuras}
\else
  \newcommand\listfigurename{Listado de Figuras}
\fi
\ifdefined\listtablename
  \renewcommand*\listtablename{Listado de Tablas}
\else
  \newcommand\listtablename{Listado de Tablas}
\fi
\ifdefined\figurename
  \renewcommand*\figurename{Figura}
\else
  \newcommand\figurename{Figura}
\fi
\ifdefined\tablename
  \renewcommand*\tablename{Tabla}
\else
  \newcommand\tablename{Tabla}
\fi
}
\@ifpackageloaded{float}{}{\usepackage{float}}
\floatstyle{ruled}
\@ifundefined{c@chapter}{\newfloat{codelisting}{h}{lop}}{\newfloat{codelisting}{h}{lop}[chapter]}
\floatname{codelisting}{Listado}
\newcommand*\listoflistings{\listof{codelisting}{Listado de Listatdos}}
\usepackage{amsthm}
\theoremstyle{definition}
\newtheorem{exercise}{Ejercicio}[chapter]
\theoremstyle{remark}
\renewcommand*{\proofname}{Prueba}
\newtheorem*{remark}{Observación}
\newtheorem*{solution}{Solución}
\makeatother
\makeatletter
\@ifpackageloaded{caption}{}{\usepackage{caption}}
\@ifpackageloaded{subcaption}{}{\usepackage{subcaption}}
\makeatother
\makeatletter
\@ifpackageloaded{tcolorbox}{}{\usepackage[many]{tcolorbox}}
\makeatother
\makeatletter
\@ifundefined{shadecolor}{\definecolor{shadecolor}{rgb}{.97, .97, .97}}
\makeatother
\makeatletter
\makeatother
\ifLuaTeX
\usepackage[bidi=basic]{babel}
\else
\usepackage[bidi=default]{babel}
\fi
\babelprovide[main,import]{spanish}
% get rid of language-specific shorthands (see #6817):
\let\LanguageShortHands\languageshorthands
\def\languageshorthands#1{}
\ifLuaTeX
  \usepackage{selnolig}  % disable illegal ligatures
\fi
\IfFileExists{bookmark.sty}{\usepackage{bookmark}}{\usepackage{hyperref}}
\IfFileExists{xurl.sty}{\usepackage{xurl}}{} % add URL line breaks if available
\urlstyle{same} % disable monospaced font for URLs
\hypersetup{
  pdftitle={Problemas de Análisis Matemático},
  pdfauthor={Alfredo Sánchez Alberca},
  pdflang={es},
  colorlinks=true,
  linkcolor={blue},
  filecolor={Maroon},
  citecolor={Blue},
  urlcolor={Blue},
  pdfcreator={LaTeX via pandoc}}

\title{Problemas de Análisis Matemático}
\author{Alfredo Sánchez Alberca}
\date{1/6/2022}

\begin{document}
\maketitle
\ifdefined\Shaded\renewenvironment{Shaded}{\begin{tcolorbox}[borderline west={3pt}{0pt}{shadecolor}, interior hidden, boxrule=0pt, enhanced, sharp corners, breakable, frame hidden]}{\end{tcolorbox}}\fi

\renewcommand*\contentsname{Indice de contenidos}
{
\hypersetup{linkcolor=}
\setcounter{tocdepth}{2}
\tableofcontents
}
\bookmarksetup{startatroot}

\hypertarget{prefacio}{%
\chapter*{Prefacio}\label{prefacio}}
\addcontentsline{toc}{chapter}{Prefacio}

Colección de problemas de Análisis Matemático aplicado.

\bookmarksetup{startatroot}

\hypertarget{teoruxeda-de-conjuntos}{%
\chapter{Teoría de conjuntos}\label{teoruxeda-de-conjuntos}}

\leavevmode\vadjust pre{\hypertarget{exr-operaciones-conjuntos}{}}%
\begin{exercise}[]\label{exr-operaciones-conjuntos}

Dado el conjunto universo de los números de un dado
\(\Omega=\{1, 2, 3, 4, 5, 6\}\) y los subconjuntos correspondientes a
sacar par en el lanzamiento de un dado \(A=\{2, 4, 6\}\) y sacar menos
de 5 en el lanzamiento de un dado \(B=\{1, 2, 3, 4\}\), calcular e
interpretar los siguientes conjuntos:

\begin{enumerate}
\def\labelenumi{\alph{enumi}.}
\tightlist
\item
  \(A\cup B\)
\item
  \(A\cap B\)
\item
  \(\overline A\) y \(\overline B\)
\item
  \(A-B\) y \(B-A\)
\item
  \(A\triangle B\)
\item
  \(\overline{(A\cup B)}\)
\item
  \(\overline{(A\cap B)}\)
\item
  \(A\cup \overline B\)
\item
  \(\overline{\overline A \cap B}\)
\end{enumerate}

¿Qué conjuntos de números en el lanzamiento de un dado serían disjuntos
con \(A\)? ¿Y con \(A\cup B\)?

\end{exercise}

\begin{tcolorbox}[enhanced jigsaw, coltitle=black, title=\textcolor{quarto-callout-tip-color}{\faLightbulb}\hspace{0.5em}{Solución}, arc=.35mm, toprule=.15mm, bottomrule=.15mm, opacitybacktitle=0.6, toptitle=1mm, leftrule=.75mm, colbacktitle=quarto-callout-tip-color!10!white, bottomtitle=1mm, titlerule=0mm, breakable, colframe=quarto-callout-tip-color-frame, colback=white, opacityback=0, rightrule=.15mm, left=2mm]

\begin{enumerate}
\def\labelenumi{\alph{enumi}.}
\tightlist
\item
  \(A\cup B = \{1, 2, 3, 4, 6\}\)
\item
  \(A\cap B = \{2, 4\}\)
\item
  \(\overline A = \{1, 3, 5\}\) y \(\overline B = \{5, 6\}\)
\item
  \(A-B = \{6\}\) y \(B-A = \{1,3\}\)
\item
  \(A\triangle B = \{1, 3, 6\}\)
\item
  \(\overline{(A\cup B)} = \{5\}\)
\item
  \(\overline{(A\cap B)} = \{1,3, 5, 6\}\)
\item
  \(A\cup \overline B = \{2, 4, 5, 6\}\)
\item
  \(\overline{\overline A \cap B} = \{2, 4, 5, 6\}\)
\end{enumerate}

Serían disjuntos con \(A\) todos los conjuntos que solo tuviesen alguno
de los números \(1\), \(3\) o \(5\), por ejemplo el conjunto
\(\{1, 5\}\). El único conjunto disjunto con \(A\cup B\), además del
vacío es \(\{5\}\).

\end{tcolorbox}

\leavevmode\vadjust pre{\hypertarget{exr-expresion-conjuntos}{}}%
\begin{exercise}[]\label{exr-expresion-conjuntos}

Expresar con operaciones entre los conjuntos \(A\), \(B\) y \(C\), los
conjuntos que se corresponden con las regiones sombreadas en los
siguientes diagramas.

\end{exercise}

\begin{figure}

\begin{minipage}[t]{0.33\linewidth}

{\centering 

\raisebox{-\height}{

\begin{venndiagram3sets}[showframe=false,shade=blue!20]
\fillOnlyB
\fillCCapB
\end{venndiagram3sets}


}

\caption{a.}

}

\end{minipage}%
%
\begin{minipage}[t]{0.33\linewidth}

{\centering 

\raisebox{-\height}{

\usepackage{venndiagram}

\begin{venndiagram3sets}[showframe=false,shade=blue!20]
\fillOnlyA
\fillOnlyC
\fillACapB
\end{venndiagram3sets}


}

\caption{b.}

}

\end{minipage}%
%
\begin{minipage}[t]{0.33\linewidth}

{\centering 

\raisebox{-\height}{

\begin{venndiagram3sets}[showframe=false,shade=blue!20]
\fillOnlyA
\fillOnlyB
\fillOnlyC
\end{venndiagram3sets}


}

\caption{b.}

}

\end{minipage}%

\end{figure}

\begin{tcolorbox}[enhanced jigsaw, coltitle=black, title=\textcolor{quarto-callout-tip-color}{\faLightbulb}\hspace{0.5em}{Solución}, arc=.35mm, toprule=.15mm, bottomrule=.15mm, opacitybacktitle=0.6, toptitle=1mm, leftrule=.75mm, colbacktitle=quarto-callout-tip-color!10!white, bottomtitle=1mm, titlerule=0mm, breakable, colframe=quarto-callout-tip-color-frame, colback=white, opacityback=0, rightrule=.15mm, left=2mm]

\begin{enumerate}
\def\labelenumi{\alph{enumi}.}
\tightlist
\item
  \((B-A)\cup (A\cap B\cap C)\)
\item
  \((A\cup B)\cap \overline{(A\cap C)}\cap \overline{(B\cap C)}\cup (A\cap B\cap C)\)
\item
  \((A\cup B\cup C) - ((A\cap B)\cup (A\cap C)\cup (B\cap C))\)
\end{enumerate}

\end{tcolorbox}

\leavevmode\vadjust pre{\hypertarget{exr-leyes-morgan}{}}%
\begin{exercise}[]\label{exr-leyes-morgan}

Demostrar gráficamente las leyes de Morgan
\(\overline{A\cup B}=\overline A \cap \overline B\) y
\(\overline{A\cap B}=\overline A \cup \overline B\).

\end{exercise}

\leavevmode\vadjust pre{\hypertarget{exr-conjunto-potencia}{}}%
\begin{exercise}[]\label{exr-conjunto-potencia}

Construir por extensión el conjunto potencia del conjunto de los grupos
sanguíneos \(S=\{0, A, B, AB\}\). ¿Cuál es su cardinal?

\end{exercise}

\begin{tcolorbox}[enhanced jigsaw, coltitle=black, title=\textcolor{quarto-callout-tip-color}{\faLightbulb}\hspace{0.5em}{Solución}, arc=.35mm, toprule=.15mm, bottomrule=.15mm, opacitybacktitle=0.6, toptitle=1mm, leftrule=.75mm, colbacktitle=quarto-callout-tip-color!10!white, bottomtitle=1mm, titlerule=0mm, breakable, colframe=quarto-callout-tip-color-frame, colback=white, opacityback=0, rightrule=.15mm, left=2mm]
\[
\begin{aligned}
\mathcal{P}(S)&=\{\emptyset, \{A\}, \{B\}, \{AB\}, \newline 
& \{\emptyset,A\},\{\emptyset,B\}, \{\emptyset,AB\}, \{A,B\}, \{A,AB\}, \{B,AB\}, \{\emptyset,A,B\},\newline
& \{\emptyset,A,AB\}, \{\emptyset,B,AB\}, \{A,B,AB\}, \newline 
& \{\emptyset,A,B,AB\} \}
\end{aligned}
\]
\end{tcolorbox}

\leavevmode\vadjust pre{\hypertarget{exr-producto-cartesiano-grupos-sanguineos}{}}%
\begin{exercise}[]\label{exr-producto-cartesiano-grupos-sanguineos}

Construir el producto cartesiano del conjunto d los grupos sanguíneos
\(S=\{0, A, B, AB\}\) y el conjunto de los factores Rh
\(R=\{\mbox{Rh}+, \mbox{Rh}-\}\).

\end{exercise}

\begin{tcolorbox}[enhanced jigsaw, coltitle=black, title=\textcolor{quarto-callout-tip-color}{\faLightbulb}\hspace{0.5em}{Solución}, arc=.35mm, toprule=.15mm, bottomrule=.15mm, opacitybacktitle=0.6, toptitle=1mm, leftrule=.75mm, colbacktitle=quarto-callout-tip-color!10!white, bottomtitle=1mm, titlerule=0mm, breakable, colframe=quarto-callout-tip-color-frame, colback=white, opacityback=0, rightrule=.15mm, left=2mm]
\[
\begin{aligned}
S \times R &=\{(0,\mbox{Rh}+), (0,\mbox{Rh}-), (A,\mbox{Rh}+)), (A,\mbox{Rh}-),\newline
& (B,\mbox{Rh}+), (B,\mbox{Rh}-), (AB,\mbox{Rh}+), (AB,\mbox{Rh}-) \}
\end{aligned}
\]
\end{tcolorbox}

\leavevmode\vadjust pre{\hypertarget{exr-relacion-equivalencia-1}{}}%
\begin{exercise}[]\label{exr-relacion-equivalencia-1}

Demostrar que la relación
\(R=\{(x,y)\in \mathbb{Z}^2: x-y \mbox{ es par}\}\) es una relación de
equivalencia.

\end{exercise}

\begin{tcolorbox}[enhanced jigsaw, coltitle=black, title=\textcolor{quarto-callout-tip-color}{\faLightbulb}\hspace{0.5em}{Solución}, arc=.35mm, toprule=.15mm, bottomrule=.15mm, opacitybacktitle=0.6, toptitle=1mm, leftrule=.75mm, colbacktitle=quarto-callout-tip-color!10!white, bottomtitle=1mm, titlerule=0mm, breakable, colframe=quarto-callout-tip-color-frame, colback=white, opacityback=0, rightrule=.15mm, left=2mm]
\emph{Propiedad reflexiva}: \(\forall a\in\mathbb{Z}\) \(a-a=0\) es par,
de manera que \(aRa\).\\
\emph{Propiedad simétrica}: \(\forall a,b\in\mathbb{Z}\) si \(aRb\)
entonces \(a-b\) es par, es decir, existe \(k\in \mathbb{Z}\) tal que
\(a-b=2k\). Por tanto, \(b-a=2(-k)\) también es par y \(bRa\).
\emph{Propiedad transitiva}: \(\forall a,b,c\in\mathbb{Z}\), si \(aRb\)
y \(bRc\) entonces \(a-b\) y \(b-c\) son pares, de manera que su suma
\(a-b+b-c = a-c\) también es par, y \(aRc\).
\end{tcolorbox}

\leavevmode\vadjust pre{\hypertarget{exr-relaciones-equivalencia}{}}%
\begin{exercise}[]\label{exr-relaciones-equivalencia}

¿Cuáles de las siguientes relaciones son relaciones de equivalencia?
¿Cuáles don de orden?

\begin{enumerate}
\def\labelenumi{\alph{enumi}.}
\tightlist
\item
  \(R_1=\{(x,y)\in \mathbb{R}^2: x = y\}\)
\item
  \(R_2=\{(x,y)\in \mathbb{R}^2: x\leq y\}\)
\item
  \(R_3=\{(x,y)\in \mathbb{R}^2: x^2 + y^2 = 1\}\)
\item
  \(R_4=\{(x,y)\in \mathbb{R}^2: x^2 + y^2 \leq 1\}\)
\end{enumerate}

\end{exercise}

\begin{tcolorbox}[enhanced jigsaw, coltitle=black, title=\textcolor{quarto-callout-tip-color}{\faLightbulb}\hspace{0.5em}{Solución}, arc=.35mm, toprule=.15mm, bottomrule=.15mm, opacitybacktitle=0.6, toptitle=1mm, leftrule=.75mm, colbacktitle=quarto-callout-tip-color!10!white, bottomtitle=1mm, titlerule=0mm, breakable, colframe=quarto-callout-tip-color-frame, colback=white, opacityback=0, rightrule=.15mm, left=2mm]

\begin{enumerate}
\def\labelenumi{\alph{enumi}.}
\tightlist
\item
  \(R_1\) es relación de equivalencia.
\item
  \(R_2\) es relación de orden.
\item
  \(R_3\) no es relación de equivalencia ni de orden porque no cumple
  las propiedades reflexiva y transitiva.
\item
  \(R_4\) es no es relación de equivalencia ni de orden porque tampoco
  cumple las propiedades reflexiva y transitiva.
\end{enumerate}

\end{tcolorbox}

\leavevmode\vadjust pre{\hypertarget{exr-supremo-infimo-maximo-minimo}{}}%
\begin{exercise}[]\label{exr-supremo-infimo-maximo-minimo}

Para cada uno de los conjuntos siguientes, calcular si existe el
supremo, el ínfimo, el máximo y el mínimo.

\begin{enumerate}
\def\labelenumi{\alph{enumi}.}
\tightlist
\item
  \(A=\{1, 2, 3, 4, 5\}\)
\item
  \(B=\{x\in\mathbb{N} : x \mbox{ es par}\}\)
\item
  \(C=\{x\in\mathbb{Q} : 0< x \leq 1\}\)
\end{enumerate}

\end{exercise}

\begin{tcolorbox}[enhanced jigsaw, coltitle=black, title=\textcolor{quarto-callout-tip-color}{\faLightbulb}\hspace{0.5em}{Solución}, arc=.35mm, toprule=.15mm, bottomrule=.15mm, opacitybacktitle=0.6, toptitle=1mm, leftrule=.75mm, colbacktitle=quarto-callout-tip-color!10!white, bottomtitle=1mm, titlerule=0mm, breakable, colframe=quarto-callout-tip-color-frame, colback=white, opacityback=0, rightrule=.15mm, left=2mm]

\begin{enumerate}
\def\labelenumi{\alph{enumi}.}
\tightlist
\item
  \(\sup(A)=5\), \(\inf(A) = 1\), \(\max(A)=5\), \(\min(A)=1\).
\item
  \(\inf(B) = 2\) y \(\min(B)=2\). No existe el supremo ni el máximo
  porque \(B\) no está acotado superiormente.
\item
  \(\sup(C)=1\), \(\inf(C) = 0\) y \(\max(C)=1\). No existe el mínimo.
\end{enumerate}

\end{tcolorbox}

\leavevmode\vadjust pre{\hypertarget{exr-ejemplos-funciones}{}}%
\begin{exercise}[]\label{exr-ejemplos-funciones}

Dar ejemplos de funciones \(f:\mathbb{Z}\rightarrow \mathbb{Z}\) que
cumplan lo siguiente:

\begin{enumerate}
\def\labelenumi{\alph{enumi}.}
\tightlist
\item
  \(f\) es inyectiva pero no sobreyectiva.
\item
  \(f\) es sobreyectiva pero no inyectiva.
\item
  \(f\) no es inyectiva ni sobreyectiva.
\item
  \(f\) es biyectiva y distinta de la función identidad.
\end{enumerate}

\end{exercise}

\begin{tcolorbox}[enhanced jigsaw, coltitle=black, title=\textcolor{quarto-callout-tip-color}{\faLightbulb}\hspace{0.5em}{Solución}, arc=.35mm, toprule=.15mm, bottomrule=.15mm, opacitybacktitle=0.6, toptitle=1mm, leftrule=.75mm, colbacktitle=quarto-callout-tip-color!10!white, bottomtitle=1mm, titlerule=0mm, breakable, colframe=quarto-callout-tip-color-frame, colback=white, opacityback=0, rightrule=.15mm, left=2mm]

\begin{enumerate}
\def\labelenumi{\alph{enumi}.}
\tightlist
\item
  \(f(x)=2x\)
\item
  \(f(x)=x^3-x\).
\item
  \(f(x)=x^2\)
\item
  \(f(x)=2x+1\)
\end{enumerate}

\end{tcolorbox}

\leavevmode\vadjust pre{\hypertarget{exr-tipos-funciones}{}}%
\begin{exercise}[]\label{exr-tipos-funciones}

Dadas las siguientes funciones de \(\mathbb{R}\) en \(\mathbb{R}\),
estudiar cuáles son inyectivas y cuáles sobreyectivas:

\begin{enumerate}
\def\labelenumi{\alph{enumi}.}
\tightlist
\item
  \(f(x)=x^2\)
\item
  \(g(x)=x^3\)
\item
  \(h(x)=x^3-x^2-2x\)
\item
  \(i(x)=|x|\)
\end{enumerate}

\end{exercise}

\begin{tcolorbox}[enhanced jigsaw, coltitle=black, title=\textcolor{quarto-callout-tip-color}{\faLightbulb}\hspace{0.5em}{Solución}, arc=.35mm, toprule=.15mm, bottomrule=.15mm, opacitybacktitle=0.6, toptitle=1mm, leftrule=.75mm, colbacktitle=quarto-callout-tip-color!10!white, bottomtitle=1mm, titlerule=0mm, breakable, colframe=quarto-callout-tip-color-frame, colback=white, opacityback=0, rightrule=.15mm, left=2mm]

\begin{enumerate}
\def\labelenumi{\alph{enumi}.}
\tightlist
\item
  \(f(x)=x^2\) no es ni inyectiva ni sobreyectiva.
\item
  \(g(x)=x^3\) es biyectiva.
\item
  \(h(x)=x^3-x^2-2x\) es sobreyectiva pero no inyectiva.
\item
  \(i(x)=|x|\) no es ni inyectiva ni sobreyectiva.
\end{enumerate}

\end{tcolorbox}

\leavevmode\vadjust pre{\hypertarget{exr-composicion-funciones-inyectivas}{}}%
\begin{exercise}[]\label{exr-composicion-funciones-inyectivas}

Demostrar que la composición de dos funciones inyectivas es también
inyectiva.

\end{exercise}

\begin{tcolorbox}[enhanced jigsaw, coltitle=black, title=\textcolor{quarto-callout-tip-color}{\faLightbulb}\hspace{0.5em}{Solución}, arc=.35mm, toprule=.15mm, bottomrule=.15mm, opacitybacktitle=0.6, toptitle=1mm, leftrule=.75mm, colbacktitle=quarto-callout-tip-color!10!white, bottomtitle=1mm, titlerule=0mm, breakable, colframe=quarto-callout-tip-color-frame, colback=white, opacityback=0, rightrule=.15mm, left=2mm]
Sean \(f\) y \(g\) dos funciones inyectivas tales que
\(\operatorname{Im}(f)\subseteq\operatorname{Dom}(g)\). Veamos que
\(g\circ f\) es inyectiva. Supongamos ahora que existen
\(a, b\in \operatorname{Dom}(f)\) tales que \(g\circ f(a)=g\circ f(b)\),
es decir, \(g(f(a))=g(f(b))\). Como \(g\) es inyectiva, se tiene que
\(f(a)=f(b)\), y como \(f\) es inyectiva se tiene que \(a=b\), con lo
que \(g\circ f\) es inyectiva.
\end{tcolorbox}

\leavevmode\vadjust pre{\hypertarget{exr-cardinal-union-producto-cartesiano}{}}%
\begin{exercise}[]\label{exr-cardinal-union-producto-cartesiano}

Dados dos conjuntos finitos \(A\) y \(B\), demostrar que
\(|A\cup B| = |A|+|B|-|A\cap B|\) y que \(|A\times B|=|A||B|\).

\end{exercise}

\begin{tcolorbox}[enhanced jigsaw, coltitle=black, title=\textcolor{quarto-callout-tip-color}{\faLightbulb}\hspace{0.5em}{Solución}, arc=.35mm, toprule=.15mm, bottomrule=.15mm, opacitybacktitle=0.6, toptitle=1mm, leftrule=.75mm, colbacktitle=quarto-callout-tip-color!10!white, bottomtitle=1mm, titlerule=0mm, breakable, colframe=quarto-callout-tip-color-frame, colback=white, opacityback=0, rightrule=.15mm, left=2mm]

\begin{enumerate}
\def\labelenumi{\alph{enumi}.}
\item
  \(A\cup B = (A-B) \cup (B-A)\cup (A\cap B)\) con \((A-B)\),
  \(A\cup B\) y \(B-A\) disjuntos dos a dos, de manera que
  \(|A\cup B| = |A-B| + |B-A| + |A\cap B|\).

  Por otro lado, \(A=(A-B)\cup (A\cap B)\), y \(B=(B-A)\cup (A\cap B)\),
  de modo que

  \[
   \begin{aligned}
   |A| + |B| - |A\cap B| &= |A-B| + |A\cap B| + |B-A| + |A\cap B| - |A\cap B| \newline
   &= |A-B| + |B-A| + |A\cap B|,
   \end{aligned}
   \] que coincide con el resultado anterior.
\item
  Supongamos que \(A=\{a_1,\ldots, a_n\}\) y \(B=\{b_1,\ldots, b_m\}\),
  de manera que \(|A|=n\) y \(|B|=m\). Para cada elemento \(a_i\in A\)
  se pueden formar \(m\) pares \((a_i,b_1),\ldots (a_i,b_m)\). Como
  \(A\) tiene \(n\) elementos, en total se pueden formar \(n\cdot m\)
  pares, así que \(|A\times B| = n\cdot m = |A||B|\).
\end{enumerate}

\end{tcolorbox}

\leavevmode\vadjust pre{\hypertarget{exr-cardinal-funcion-inyectiva-sobreyectiva}{}}%
\begin{exercise}[]\label{exr-cardinal-funcion-inyectiva-sobreyectiva}

Dada una función \(f:A\rightarrow B\), demostrar que si \(f\) es
inyectiva, entonces \(|A|\leq |B|\), y si \(f\) es sobreyectiva,
entonces \(|A|\geq |B|\). ¿Cómo es \(|A|\) en comparación con \(|B|\)
cuando \(f\) es biyectiva?

\end{exercise}

\begin{tcolorbox}[enhanced jigsaw, coltitle=black, title=\textcolor{quarto-callout-tip-color}{\faLightbulb}\hspace{0.5em}{Solución}, arc=.35mm, toprule=.15mm, bottomrule=.15mm, opacitybacktitle=0.6, toptitle=1mm, leftrule=.75mm, colbacktitle=quarto-callout-tip-color!10!white, bottomtitle=1mm, titlerule=0mm, breakable, colframe=quarto-callout-tip-color-frame, colback=white, opacityback=0, rightrule=.15mm, left=2mm]
Sea \(f:A\rightarrow B\) inyectiva. Entonces para cualesquiera
\(a_1,a_2\in A\) con \(a_1\neq a_2\) se tiene que \(f(a_1)\neq f(a_2)\),
por lo que \(|A|\leq |B|\).

Sea \(f:A\rightarrow B\) sobreyectiva. Entonces para todo \(b\in B\)
existe \(a\in A\) tal que \(f(a)=b\). Además dos elementos de \(B\) no
pueden tener la misma preimagen porque entonces \(f\) no sería una
función, por lo que \(|A|\geq |B|\).

De lo anterior se deduce que si \(f\) es biyectiva, entonces
\(|A|=|B|\).
\end{tcolorbox}

\leavevmode\vadjust pre{\hypertarget{exr-numero-funciones-inyectivas}{}}%
\begin{exercise}[]\label{exr-numero-funciones-inyectivas}

Dados dos conjuntos finitos \(A\) y \(B\) con \(|A|=n\) y \(|B|=m\).
¿Cuántas funciones distintas se pueden construir de \(A\) a \(B\). ¿Y
cuántas funciones inyectivas suponiendo que \(n\leq m\)?

\end{exercise}

\begin{tcolorbox}[enhanced jigsaw, coltitle=black, title=\textcolor{quarto-callout-tip-color}{\faLightbulb}\hspace{0.5em}{Solución}, arc=.35mm, toprule=.15mm, bottomrule=.15mm, opacitybacktitle=0.6, toptitle=1mm, leftrule=.75mm, colbacktitle=quarto-callout-tip-color!10!white, bottomtitle=1mm, titlerule=0mm, breakable, colframe=quarto-callout-tip-color-frame, colback=white, opacityback=0, rightrule=.15mm, left=2mm]
Se pueden construir \(m^n\) funciones distintas, y \(\frac{m!}{(m-n)!}\)
funciones inyectivas.
\end{tcolorbox}

\leavevmode\vadjust pre{\hypertarget{exr-ejemplo-conjunto-infinito-complemento}{}}%
\begin{exercise}[]\label{exr-ejemplo-conjunto-infinito-complemento}

Tomando el conjunto de los números naturales \(\mathbb{N}\) como
conjunto universo, dar un ejemplo de un subconjunto infinito cuyo
complemento también sea infinito.

\end{exercise}

\begin{tcolorbox}[enhanced jigsaw, coltitle=black, title=\textcolor{quarto-callout-tip-color}{\faLightbulb}\hspace{0.5em}{Solución}, arc=.35mm, toprule=.15mm, bottomrule=.15mm, opacitybacktitle=0.6, toptitle=1mm, leftrule=.75mm, colbacktitle=quarto-callout-tip-color!10!white, bottomtitle=1mm, titlerule=0mm, breakable, colframe=quarto-callout-tip-color-frame, colback=white, opacityback=0, rightrule=.15mm, left=2mm]
\(A=\{x\in\mathbb{N}: x \mbox{ es par}\}\) es infinito y
\(\overline A=\{x\in\mathbb{N}: x \mbox{ es impar}\}\) también es
infinito.
\end{tcolorbox}

\leavevmode\vadjust pre{\hypertarget{exr-subconjunto-infinito-numerable}{}}%
\begin{exercise}[]\label{exr-subconjunto-infinito-numerable}

Demostrar que todo conjunto infinito tiene un subconjunto infinito
numerable.

\end{exercise}

\begin{tcolorbox}[enhanced jigsaw, coltitle=black, title=\textcolor{quarto-callout-tip-color}{\faLightbulb}\hspace{0.5em}{Solución}, arc=.35mm, toprule=.15mm, bottomrule=.15mm, opacitybacktitle=0.6, toptitle=1mm, leftrule=.75mm, colbacktitle=quarto-callout-tip-color!10!white, bottomtitle=1mm, titlerule=0mm, breakable, colframe=quarto-callout-tip-color-frame, colback=white, opacityback=0, rightrule=.15mm, left=2mm]
Sean \(A\) un conjunto infinito. Como \(A\) no es vacío, existe un
elemento \(a_1\in A\). Considérese ahora el conjunto
\(A_1 = A\setminus \{a_1\}\). Es evidente que \(A_1\) sigue siendo
infinito y podemos elegir otro elemento \(a_2\in A_1\) de manera que el
conjunto \(A_2=A_1-\{a_2\}\) sigue siendo infinito. Repitiendo este
proceso indefinidamente obtenemos que el conjunto
\(\{a_1, a_2, \ldots\}\) es un subconjunto de \(A\) que es numerable.
\end{tcolorbox}

\leavevmode\vadjust pre{\hypertarget{exr-conjunto-equipotente-subconjunto}{}}%
\begin{exercise}[]\label{exr-conjunto-equipotente-subconjunto}

Demostrar que un conjunto es infinito si y solo si es equipotente a un
subconjunto propio.

\end{exercise}

\begin{tcolorbox}[enhanced jigsaw, coltitle=black, title=\textcolor{quarto-callout-tip-color}{\faLightbulb}\hspace{0.5em}{Solución}, arc=.35mm, toprule=.15mm, bottomrule=.15mm, opacitybacktitle=0.6, toptitle=1mm, leftrule=.75mm, colbacktitle=quarto-callout-tip-color!10!white, bottomtitle=1mm, titlerule=0mm, breakable, colframe=quarto-callout-tip-color-frame, colback=white, opacityback=0, rightrule=.15mm, left=2mm]
Sea \(A\) un conjunto. Si \(A\) es finito, entonces cualquier
subconjunto \(B\subset A\) cumple que \(|B| < |A|\) por lo que no se
puede establecer una biyección entre \(A\) y \(B\).

Si \(A\) es infinito, por el ejercicio anterior se tiene que existe un
subconjunto numerable \(B=\{a_1,a_2,\ldots\}\subseteq A\). Si tomamos la
aplicación \(f:B\to B\setminus\{a_1\}\) dada por \(f(a_i)=a_{i+1}\),
entonces \(f\) es biyectiva, y su extensión
\(\hat f: A\to A\setminus\{a_1\}\) dada por

\[
\hat f(x)=
\begin{cases}
x & \mbox{si } x\not\in B\\
f(x) & \mbox{si } x\in B
\end{cases}
\]

es también biyectiva, por lo que \(A\) es equipotente a
\(A\setminus \{a_1\}\) que es un subconjunto propio suyo.
\end{tcolorbox}

\leavevmode\vadjust pre{\hypertarget{exr-producto-cartesiano-numerable}{}}%
\begin{exercise}[]\label{exr-producto-cartesiano-numerable}

Demostrar que el producto cartesiano de dos conjuntos numerables es
numerable. ¿Y el producto cartesiano de \(n\) conjuntos numerables?

\end{exercise}

\begin{tcolorbox}[enhanced jigsaw, coltitle=black, title=\textcolor{quarto-callout-tip-color}{\faLightbulb}\hspace{0.5em}{Solución}, arc=.35mm, toprule=.15mm, bottomrule=.15mm, opacitybacktitle=0.6, toptitle=1mm, leftrule=.75mm, colbacktitle=quarto-callout-tip-color!10!white, bottomtitle=1mm, titlerule=0mm, breakable, colframe=quarto-callout-tip-color-frame, colback=white, opacityback=0, rightrule=.15mm, left=2mm]
Sean \(A\) y \(B\) dos conjuntos numerables. Entonces existe una
aplicación inyectiva \(f:A\to\mathbb{N}\) y otra \(g:B\to\mathbb{N}\).
Si se toma ahora la función \(h:A\times B\to \mathbb{N}\) definida como

\[ f(a,b) = 2^{f(a)}3^{g(b)}\, \forall a\in A, b\in B,\]

se tiene que \(f\) es inyectiva y por tanto \(A\times B\) es numerable.

Por inducción, es fácil probar que el producto cartesiano de \(n\)
conjuntos numerables es también numerable.
\end{tcolorbox}

\leavevmode\vadjust pre{\hypertarget{exr-racionales-numerables}{}}%
\begin{exercise}[]\label{exr-racionales-numerables}

Demostrar que el conjunto de los números racionales es numerable.

\end{exercise}

\begin{tcolorbox}[enhanced jigsaw, coltitle=black, title=\textcolor{quarto-callout-tip-color}{\faLightbulb}\hspace{0.5em}{Solución}, arc=.35mm, toprule=.15mm, bottomrule=.15mm, opacitybacktitle=0.6, toptitle=1mm, leftrule=.75mm, colbacktitle=quarto-callout-tip-color!10!white, bottomtitle=1mm, titlerule=0mm, breakable, colframe=quarto-callout-tip-color-frame, colback=white, opacityback=0, rightrule=.15mm, left=2mm]
Si se considera la aplicación
\(f:\mathbb{Q}\to \mathbb{Z}\times \mathbb{N}\) que a cada número
racional \(r\) le hace corresponder el par
\((p,q)\in \mathbb{Z}\times \mathbb{N}\) donde \(\frac{p}{q}\) es la
fracción irreducible de \(r\) con denominador positivo, se tiene que
\(f\) es inyectiva. Como el producto cartesiano de dos conjuntos
numerables es numerable, existe otra aplicación inyectiva de
\(g:\mathbb{Z}\times \mathbb{N}\to \mathbb{N}\), con lo que
\(g\circ f:\mathbb{Q}\to\mathbb{N}\) es inyectiva y \(\mathbb{Q}\) es
numerable.
\end{tcolorbox}

\leavevmode\vadjust pre{\hypertarget{exr-union-numerables}{}}%
\begin{exercise}[]\label{exr-union-numerables}

Demostrar que la unión de dos conjuntos numerables es numerable.

\end{exercise}

\begin{tcolorbox}[enhanced jigsaw, coltitle=black, title=\textcolor{quarto-callout-tip-color}{\faLightbulb}\hspace{0.5em}{Solución}, arc=.35mm, toprule=.15mm, bottomrule=.15mm, opacitybacktitle=0.6, toptitle=1mm, leftrule=.75mm, colbacktitle=quarto-callout-tip-color!10!white, bottomtitle=1mm, titlerule=0mm, breakable, colframe=quarto-callout-tip-color-frame, colback=white, opacityback=0, rightrule=.15mm, left=2mm]
Sean \(A\) y \(B\) dos conjuntos numerables disjuntos. Entonces existen
dos biyecciones \(f:A\to \mathbb{N}\) y \(g:B\to \mathbb{N}\). A partir
de estas biyecciones se puede definir otra \(h:A\cup B\to \mathbb{N}\)
dada por

\[h(x)=
\begin{cases}
2f(x)-1 & \mbox{si } x\in A\\
2g(x) & \mbox{si } x\in B
\end{cases}
\]

Así pues, \(A\cup B\) es numerable.

Si \(A\) y \(B\) no son disjuntos, entonces
\(A\cup B=A\cup (B\setminus A)\). Si
\(B\setminus A=\{b_1,\ldots, b_n\}\) es finito, se puede tomar la
biyección \(g=\{(b_1,1),\ldots,(b_n,n)\}\) y, a partir de ella,
construir la biyección \(h:A\cup B\to \mathbb{N}\) dada por

\[h(x)=
\begin{cases}
f(x)+n & \mbox{si } x\in A\\
g(x) & \mbox{si } x\in B\setminus A
\end{cases}
\]

Mientras que si \(B\setminus A\) es infinito, se puede razonar como al
principio pues \(A\) y \(B\setminus A\) son disjuntos.
\end{tcolorbox}

\leavevmode\vadjust pre{\hypertarget{exr-irracionales-no-numerables}{}}%
\begin{exercise}[]\label{exr-irracionales-no-numerables}

Demostrar que el conjunto de los números irracionales no es numerable.

\end{exercise}

\begin{tcolorbox}[enhanced jigsaw, coltitle=black, title=\textcolor{quarto-callout-tip-color}{\faLightbulb}\hspace{0.5em}{Solución}, arc=.35mm, toprule=.15mm, bottomrule=.15mm, opacitybacktitle=0.6, toptitle=1mm, leftrule=.75mm, colbacktitle=quarto-callout-tip-color!10!white, bottomtitle=1mm, titlerule=0mm, breakable, colframe=quarto-callout-tip-color-frame, colback=white, opacityback=0, rightrule=.15mm, left=2mm]
Ya hemos visto en el ejercicio Ejercicio~\ref{exr-racionales-numerables}
que \(\mathbb{Q}\) es numerable, de manera que si
\(\mathbb{R}\setminus \mathbb{Q}\) fuese numerable, entonces por el
Ejercicio~\ref{exr-union-numerables}
\(\mathbb{Q}\cup \mathbb{R}\setminus \mathbb{Q}=\mathbb{R}\) sería
numerable, lo cual no es cierto.
\end{tcolorbox}

\leavevmode\vadjust pre{\hypertarget{exr-union-numerable-conjuntos-numerables}{}}%
\begin{exercise}[]\label{exr-union-numerable-conjuntos-numerables}

Demostrar la unión de un conjunto numerable de conjuntos numerables es
numerable.

\end{exercise}

\begin{tcolorbox}[enhanced jigsaw, coltitle=black, title=\textcolor{quarto-callout-tip-color}{\faLightbulb}\hspace{0.5em}{Solución}, arc=.35mm, toprule=.15mm, bottomrule=.15mm, opacitybacktitle=0.6, toptitle=1mm, leftrule=.75mm, colbacktitle=quarto-callout-tip-color!10!white, bottomtitle=1mm, titlerule=0mm, breakable, colframe=quarto-callout-tip-color-frame, colback=white, opacityback=0, rightrule=.15mm, left=2mm]
Sea \(A\) un conjunto numerable de conjuntos numerables. Por ser \(A\)
numerable existe una biyección \(f:\mathbb{N}\to A\), de manera que
podemos enumerar los elementos de \(A\) de tal forma que \(A_i=f(i)\).
Del mismo modo, como cada conjunto \(A_i\) es numerable se puede
establecer una enumeración de sus elementos
\(A_i=\{a_{i1},a_{i2},\ldots\}\). Así pues, podemos representar los
elementos de \(\cup_{i=1}^\infty A_i\) en una tabla como la siguiente

\[
\begin{array}{cccccc}
a_{11} & \rightarrow & a_{12} & & a_{13} & \ldots \\
& \swarrow & & \nearrow & \downarrow \\
a_{21} & & a_{22} & & a_{23} & \ldots \\
\downarrow & \nearrow & & \swarrow \\
a_{31} & & a_{32} & & a_{33} & \ldots \\
\vdots & & \vdots & & \vdots & \ddots
\end{array}
\]

Siguiendo el orden de las flechas es posible enumerar todos los
elementos de este conjunto, por lo que \(\cup_{i=1}^\infty A_i\) es
numerable.
\end{tcolorbox}

\leavevmode\vadjust pre{\hypertarget{exr-conjunto-polinomios-no-numerable}{}}%
\begin{exercise}[]\label{exr-conjunto-polinomios-no-numerable}

Demostrar que el conjunto de todos los polinomios con coeficientes
enteros
\(P=\{a_0+a_1x+a_2x^2+\cdots+a_nx^n: n\in \mathbb{N}, a_i\in \mathbb{Z}\}\)
es numerable. ¿Y el de los polinomios con coeficientes racionales?

\end{exercise}

\begin{tcolorbox}[enhanced jigsaw, coltitle=black, title=\textcolor{quarto-callout-tip-color}{\faLightbulb}\hspace{0.5em}{Solución}, arc=.35mm, toprule=.15mm, bottomrule=.15mm, opacitybacktitle=0.6, toptitle=1mm, leftrule=.75mm, colbacktitle=quarto-callout-tip-color!10!white, bottomtitle=1mm, titlerule=0mm, breakable, colframe=quarto-callout-tip-color-frame, colback=white, opacityback=0, rightrule=.15mm, left=2mm]
Para cada \(n\in\mathbb{N}\) sea \(P_n\) el conjunto de los polinomios
de grado \(n\) con coeficientes enteros
\(P_n=\{a_0+a_1x+a_2x^2+\cdots+a_nx^n: a_i\in \mathbb{Z}\}\). Para cada
polinomio \(a_0+a_1x+a_2x^2+\cdots+a_nx^n\in P_n\) podemos establecer
una biyección entre sus coeficientes y la tupla
\(p_n=(a_0,a_1,\ldots,a_n)\), con \(a_i\in\mathbb{Z}\). Por tanto,
existe una biyección entre \(P_n\) y \(\mathbb{Z}^n\), y como
\(\mathbb{Z}^n\) es numerable, el \(P_n\) también lo es.

Finalmente, \(P=\cup_{i=1}^\infty P_n\) que es la unión numerable de
conjuntos numerables, que, como ya se vió en el
Ejercicio~\ref{exr-union-numerable-conjuntos-numerables}, es numerable.
\end{tcolorbox}

\leavevmode\vadjust pre{\hypertarget{exr-conjuntos-numerables}{}}%
\begin{exercise}[]\label{exr-conjuntos-numerables}

¿Cuáles de los siguientes conjuntos son numerables?

\begin{enumerate}
\def\labelenumi{\alph{enumi}.}
\tightlist
\item
  \(A=\{3k: k\in \mathbb{Z}\}\)
\item
  \(B=\{x\in \mathbb{Q}: -10 < x < 10\}\)
\item
  \(C = \{x\in \mathbb{R}: 0\leq x\leq 1\}\)
\item
  \(D=\{(x,y): x\in \mathbb{Z}, y\in \mathbb{Q}\}\)
\item
  \(E=\{1/n : n\in \mathbb{N}\}\)
\end{enumerate}

\end{exercise}

\leavevmode\vadjust pre{\hypertarget{exr-conjunto-secuencias-adn-numerable}{}}%
\begin{exercise}[]\label{exr-conjunto-secuencias-adn-numerable}

¿Es el conjunto de todas las secuencias infinitas de ADN numerable?

\end{exercise}

\bookmarksetup{startatroot}

\hypertarget{nuxfameros-reales}{%
\chapter{Números reales}\label{nuxfameros-reales}}

\leavevmode\vadjust pre{\hypertarget{exr-propiedad-arquimediana-1}{}}%
\begin{exercise}[]\label{exr-propiedad-arquimediana-1}

Usando la propiedad arquimediana de los números reales, demostrar que
para cualquier número real \(a\in\mathbb{R}\) con \(a>0\) existe un
número natural \(n\) tal que \(n-1\leq a< n\).

\end{exercise}

\begin{tcolorbox}[enhanced jigsaw, coltitle=black, title=\textcolor{quarto-callout-tip-color}{\faLightbulb}\hspace{0.5em}{Solución}, arc=.35mm, toprule=.15mm, bottomrule=.15mm, opacitybacktitle=0.6, toptitle=1mm, leftrule=.75mm, colbacktitle=quarto-callout-tip-color!10!white, bottomtitle=1mm, titlerule=0mm, breakable, colframe=quarto-callout-tip-color-frame, colback=white, opacityback=0, rightrule=.15mm, left=2mm]
Como \(a>0\) se tiene que \(\frac{1}{a}>0\), y por la propiedad
arquimediana se cumple que existe \(n\in \mathbb{N}\) tal que \(x<n\).

Considérese ahora el conjunto \(A=\{m\in \mathbb{N}: x<m\}\). Como
\(x<n\) se tiene que \(x\in A\) y por tanto \(A\) no está vacío.
Aplicando ahora el principio de buena ordenación de los números
naturales, como \(A\subset \mathbb{N}\), existe un primer elemento
\(n_0\in A\), tal que \(n_0-1\not\in A\), de manera que \(n_0-1\leq a\)
y con ello se tiene que \$\(n_0-1\leq a<n\).
\end{tcolorbox}

\leavevmode\vadjust pre{\hypertarget{exr-densidad-racionales}{}}%
\begin{exercise}[]\label{exr-densidad-racionales}

Se dice que un conjunto \(A\) es denso en \(\mathbb{R}\) si cada
intervalo \((a,b)\) de \(\mathbb{R}\) contiene algún elemento de \(A\).
Demostrar que \(\mathbb{Q}\) es denso en \(\mathbb{R}\).

\end{exercise}

\begin{tcolorbox}[enhanced jigsaw, coltitle=black, title=\textcolor{quarto-callout-tip-color}{\faLightbulb}\hspace{0.5em}{Solución}, arc=.35mm, toprule=.15mm, bottomrule=.15mm, opacitybacktitle=0.6, toptitle=1mm, leftrule=.75mm, colbacktitle=quarto-callout-tip-color!10!white, bottomtitle=1mm, titlerule=0mm, breakable, colframe=quarto-callout-tip-color-frame, colback=white, opacityback=0, rightrule=.15mm, left=2mm]
La prueba es la misma que la de la propiedad arquimediana. Basta con
tomar \(n\in \mathbb{N}\) tal que \(\frac{1}{n}< b-a\). Si ahora se toma
el primer múltiplo de \(1/n\) tal que \(a<\frac{m}{n}\), también se
cumplirá que \(\frac{m}{n}<b\), ya que de lo contrario
\(\frac{m-1}{n}<a<b<\frac{m}{n}\) lo que lleva a la contradicción de que
\(\frac{1}{n}>b-a\).
\end{tcolorbox}

\leavevmode\vadjust pre{\hypertarget{exr-existencia-raiz-cuadrada}{}}%
\begin{exercise}[]\label{exr-existencia-raiz-cuadrada}

Dado un número real con \(a\in\mathbb{R}\), demostrar que existe un
número real \(x\) tal que \(x>0\) y \(x^2=a\).

\end{exercise}

\bookmarksetup{startatroot}

\hypertarget{topologuxeda-de-los-nuxfameros-reales}{%
\chapter{Topología de los números
reales}\label{topologuxeda-de-los-nuxfameros-reales}}

\leavevmode\vadjust pre{\hypertarget{exr-sucesion-intervalos-anidados}{}}%
\begin{exercise}[]\label{exr-sucesion-intervalos-anidados}

Dada la sucesión de intervalos anidados \(I_n=[0,\frac{1}{n}]\),
\(n\in\mathbb{N}\), demostrar que \(\cap_{n=1}^\infty = \{0\}\).
Demostrar también que si se consideran intervalos abiertos en lugar de
cerrados entonces la intersección es vacía.

\end{exercise}

\leavevmode\vadjust pre{\hypertarget{exr-supremo-infimo-maximo-minimo-2}{}}%
\begin{exercise}[]\label{exr-supremo-infimo-maximo-minimo-2}

Calcular el supremo y el ínfimo de los siguientes conjuntos. ¿Tienen
máximo y mínimo?

\begin{enumerate}
\def\labelenumi{\alph{enumi}.}
\tightlist
\item
  \(A=\{x\in \mathbb{R} : 2 < x^2-2 < 3\}\)
\item
  \(B=\{x\in \mathbb{R} : 1 < 4x^2 - 3 \leq 5\}\)
\end{enumerate}

\end{exercise}

\leavevmode\vadjust pre{\hypertarget{exr-interior-conjunto-unitario}{}}%
\begin{exercise}[]\label{exr-interior-conjunto-unitario}

¿Cuál es el interior del conjunto \(A=\{a\}\)?

\end{exercise}

\leavevmode\vadjust pre{\hypertarget{exr-interior-exterior-frontera}{}}%
\begin{exercise}[]\label{exr-interior-exterior-frontera}

Sean \(a,b,c\in\mathbb{R}\) tales que \(a<b<c\) y sea
\(A=\{a\}\cup (b,c)\). Calcular \(\operatorname{Int}(A)\),
\(\operatorname{Ext}(A)\) y \(\operatorname{Fr}(A)\).

\end{exercise}

\leavevmode\vadjust pre{\hypertarget{exr-interior-racionales-irracionales}{}}%
\begin{exercise}[]\label{exr-interior-racionales-irracionales}

Demostrar que el conjunto de los números racionales no tiene puntos
interiores. ¿Y el conjunto de los números irracionales?

\end{exercise}

\leavevmode\vadjust pre{\hypertarget{exr-interior-subconjunto}{}}%
\begin{exercise}[]\label{exr-interior-subconjunto}

Demostrar que si \(x\) es un punto interior de \(A\) y \(A\subseteq B\),
entonces \(x\) también es un punto interior de \(B\).

\end{exercise}

\leavevmode\vadjust pre{\hypertarget{exr-punto-interior-union-interseccion}{}}%
\begin{exercise}[]\label{exr-punto-interior-union-interseccion}

Demostrar que si \(x\) es un punto interior de dos conjuntos \(A\) y
\(B\), entonces también es un punto interior de su unión y su
intersección.

\end{exercise}

\leavevmode\vadjust pre{\hypertarget{exr-interior-union-interseccion}{}}%
\begin{exercise}[]\label{exr-interior-union-interseccion}

Demostrar que si \(A,B\subseteq \mathbb{R}\),
\(\operatorname{Int}(A\cap B) = \operatorname{Int}(A) \cap \operatorname{Int}(B)\).

Demostrar también que el anterior resultado no es cierto para la unión,
es decir, dados \(A,B\subseteq \mathbb{R}\), no se cumple siempre que
\(\operatorname{Int}(A\cup B) = \operatorname{Int}(A) \cup \operatorname{Int}(B)\).

\end{exercise}

\leavevmode\vadjust pre{\hypertarget{exr-particion-interior-exterior-frontera}{}}%
\begin{exercise}[]\label{exr-particion-interior-exterior-frontera}

Dado un conjunto \(A\subset\mathbb{R}\), probar que los conjuntos
\(\operatorname{Int}(A)\), \(\operatorname{Ext}(A)\) y
\(\operatorname{Fr}(A)\) forman una partición de \(\mathbb{R}\).

\end{exercise}

\leavevmode\vadjust pre{\hypertarget{exr-ejemplo-conjuntos-abiertos-interseccion-no-abierta}{}}%
\begin{exercise}[]\label{exr-ejemplo-conjuntos-abiertos-interseccion-no-abierta}

Dar un ejemplo de dos conjuntos no abiertos pero cuya intersección es
abierta.

\end{exercise}

\leavevmode\vadjust pre{\hypertarget{exr-union-interseccion-coleccion-abiertos-cerrados}{}}%
\begin{exercise}[]\label{exr-union-interseccion-coleccion-abiertos-cerrados}

Probar las siguientes propiedades:

\begin{enumerate}
\def\labelenumi{\alph{enumi}.}
\tightlist
\item
  La unión de una colección de conjuntos abiertos es un conjunto
  abierto.
\item
  La intersección de una colección finita de conjuntos abiertos es un
  conjunto abierto.
\item
  La intersección de una colección de conjuntos cerrados es cerrada.
\item
  La unión de una colección finita de conjuntos cerrados es un conjunto
  cerrado.
\end{enumerate}

\end{exercise}



\end{document}
